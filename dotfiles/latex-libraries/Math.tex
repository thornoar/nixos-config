\usepackage{amsmath, amsthm, amssymb, euscript, MnSymbol}

%разное
\newcommand*{\bas}[1]{\operator{\beta}{(#1)}}
\renewcommand*{\dim}[1]{\operator{dim}{#1}}
\newcommand{\ax}[2]{\textbf{(#1-#2)}}
\newcommand*{\sig}[1]{\operator{sign}{#1}}

%\newcommand{\A}{\mathcal{A}}
\newcommand{\B}{\EuScript{B}}
\newcommand{\A}{\EuScript{A}}

\binoppenalty=10000
\relpenalty=10000

\newcommand{\mc}[1]{\mathcal{#1}}
\newcommand{\mf}[1]{\mathfrak{#1}}
\newcommand{\mb}[1]{\mathbb{#1}}

\newcommand{\lr}[1]{\left( #1 \right)}

\newcommand{\tto}[1]{\underset{x \to #1}{\longrightarrow}}

\newcommand{\dd}{\mathinner{\mkern2mu\raise1pt\hbox{.}\mkern2mu \raise4pt\hbox{.}\mkern2mu\raise7pt\hbox{.}\mkern1mu}}
\newcommand{\test}{\alpha}
% \newcommand{\B}{\mathcal{B}}

\newcommand{\ov}[1]{\overline{#1}}

\newcommand{\newint}[3]{\int\limits_{#1}^{#2} #3 dx}

\newcommand*{\Equ}{$\ \Longleftrightarrow \ $}
\newcommand*{\equ}{\ \Longleftrightarrow \ }
\newcommand*{\Implic}{$\ \Longrightarrow \ $}
\newcommand*{\implic}{\ \Longrightarrow \ }

\newcommand{\tp}[1]{\langle #1 \rangle}

\newcommand{\s}[2]{\sum\limits_{#1}^{#2}}
\newcommand{\newcup}[2]{\bigcup\limits_{#1}^{#2}}
\newcommand{\newcap}[2]{\bigcap\limits_{#1}^{#2}}
\newcommand{\lle}{\leqslant}
\newcommand{\gge}{\geqslant}

\usepackage{MnSymbol}

\usepackage{mathrsfs}

\newcommand{\restrspace}[3]{
    \begin{array}[t]{c|l}
        #1 & \vphantom{\int\limits_{n}} #3  \\
         & #2 
    \end{array} 
}

\newcommand{\restr}[2]{
    \begin{array}[t]{c|c}
        #1 & \vphantom{\int\limits_{n}}  \\
         & \lefteqn{#2} 
    \end{array} 
}

\newcommand{\restrmid}[2]{
    \begin{array}{c|c}
        #1 & \\
         & #2 
    \end{array}
}

\newcommand{\xvee}{\stackrel{\displaystyle \vee}{\smash{\vee}}}

\newcommand{\ang}{
    \begin{picture}(11,7)
        \put(0,0){\line(5,4){10}}
        \put(0,0){\line(1,0){10}}
    \end{picture}
}

%---------
%Операторы
%---------

%нотация унарных операторов
\newcommand*{\operator}[2]{\operatorname{#1} #2} 
\newcommand*{\operatorIndex}[2]{\operatorname{#1}_{#2}} 
\newcommand*{\operatorTogether}[2]{#1#2}


%инфиксная нотация бинарных операторов и отношений
\newcommand*{\binOperator}[3]{#2#1#3}
\newcommand*{\binRelation}[3]{#2#1#3}
\newcommand*{\binOperatorIndex}[3]{\operatorname{#1}_{#2}#3}

%-----------------------
%Общекатегорные операторы
%-----------------------

%классы объектов и морфизмов
\newcommand{\Ob}[1]{\operator{Ob}{#1}}
\newcommand{\Mor}[1]{\operator{Mor}{#1}}

%сигнатурные категорные операторы
\newcommand*{\cod}[1]{\operator{cod}{#1}}
\newcommand*{\dom}[1]{\operator{dom}{#1}}
\newcommand*{\id}[1]{\operatorIndex{id}{#1}}
\newcommand{\comp}[2]{\binOperator{\circ}{#1}{#2}} 

\newcommand*{\codC}[2]{\operator{cod}_{#2}{#1}}
\newcommand*{\domC}[2]{\operator{dom}_{#2}{#1}}


%множество морфизмов, группа автоморфизмов, моноид эндморфизмов
\newcommand*{\Hom}[1]{\operator{Hom}{#1}}
\newcommand*{\Aut}[1]{\operator{Aut}{#1}}
\newcommand*{\End}[1]{\operator{End}{#1}}

\newcommand*{\HomC}[2]{\operator{Hom}_{#2}{#1}}
\newcommand*{\AutC}[2]{\operator{Aut}_{#2}{#1}}
\newcommand*{\EndC}[2]{\operator{End}_{#2}{#1}}

%изоморфность и каноническая изоморфность
\newcommand*{\isomorphic}[2]{\binRelation{\cong}{#1}{#2}}
\newcommand*{\cannonicIsomorphic}[2]{\binRelation{\simeq}{#1}{#2}} 

%предел диаграммы
\newcommand*{\Lim}[1]{\operator{Lim}{#1}}
\newcommand*{\LimC}[2]{\operator{Lim}_{#2}{#1}}

%-----------------
%Именные категории
%-----------------
\newcommand*{\acategoryname}[1]{\mathcal{#1}}
\newcommand*{\thecategoryname}[1]{\mathrm{#1}}
\newcommand*{\thebfcategoryname}[1]{\mathbf{#1}}

\newcommand*{\Ab}{\thecategoryname{Ab}}
\newcommand*{\Set}{\thecategoryname{Set}}
\newcommand*{\LMod}[1]{#1\operatorname{-}\thecategoryname{Mod}}
\newcommand*{\ModR}[1]{\thecategoryname{Mod}\operatorname{-}#1}
\newcommand*{\Top}{\thecategoryname{Top}}
\newcommand*{\ChainComplex}{\thecategoryname{ChainComplex}}
\newcommand*{\Cat}{\thecategoryname{Cat}}

\newcommand{\catname}[1]{\thecategoryname{#1}}


%--------------------------
%Подобъекты и факторобъекты
%--------------------------

%фактор по отношению эквивалентности #1/#2 
\newcommand*{\quotientByRelation}[2]{\binOperator{\slash}{#1}{#2}}

%фактор по подобъекту #1/#2 
\newcommand*{\quotientBySubobject}[2]{\binOperator{\slash}{#1}{#2}}

%Генераторы структур

%сигнатура структуры
\newcommand*{\struct}[1]{\langle\forcsvlist{#1}\rangle}

%--------------------
%Аддитивные категории
%--------------------

%образ и ядро
\renewcommand*{\ker}[1]{\operator{Ker}{#1}}% уже определено
\newcommand*{\coker}[1]{\operator{Coker}{#1}}
\newcommand*{\image}[1]{\operator{Im}{#1}}
\newcommand*{\im}[1]{\image{#1}}
\newcommand*{\coim}[1]{\operator{Coim}{#1}}

%--------------------
%Генераторы категорий
%--------------------

\newcommand*{\Func}[1]{\operator{Func}{#1}}
\newcommand*{\op}[1]{#1^{op}}

%--------------------
%Функторы в жизни
%--------------------

\newcommand*{\functor}[2]{\mathrm{#1}#2}

\newcommand*{\Subgroup}[1]{\functor{Subgroup}{#1}}
\newcommand*{\Cover}[1]{\functor{Cover}{#1}}
\newcommand*{\Symm}[1]{\functor{Symm}{#1}}
\newcommand*{\Bij}[1]{\functor{Bij}{#1}}


%---------------------
%Конструкторы множеств
%---------------------

%множество всех элементов таких, что
\newcommand*{\setOfAllSuchThat}[2]{\{ \, #1 \mid #2 \, \}}
%\newcommand*{\setOfAllSuchThat}[2]{\{  #1 \mid #2  \}}

\newcommand*{\setOf}[2]{\setOfAllSuchThat{#1}{#2}}

%морфизмы из A в B
\newcommand*{\fromDomToCod}[2]{#1 \to #2}

%---------------------
%Конструкторы функций
%---------------------

%морфизм данной сигнатуры
\newcommand*{\morphism}[3]{#1\colon\fromDomToCod{#2}{#3}}

%анонимная функция
\newcommand*{\mapsTo}[2]{#1 \mapsto #2}

%морфизм данной сигнатуры и реализации
\newcommand*{\morphismMapsTo}[5]{\morphism{#1}{#2}{#3}\colon #4 \mapsto #5}

%элемент данного множества
%\newcommand*{\element}[2]{#1 \in #2}


%------------------
%---Теория групп---
%------------------

%задание образующими и соотношениями <#1|#2>
\newcommand*{\presentationOf}[2]{\langle \, \forcsvlist{#1} \,\mid\, \forcsvlist{#2} \,\rangle}
\renewcommand*{\deg}[2]{\operatorIndex{deg}{#1}{#2}}

%порядок элемента
\newcommand*{\ord}[1]{\operator{ord}{#1}}

%группа внутренних автоморфизмов и группа внешних автоморфизмов
\newcommand{\Inn}[1]{\operator{Inn}{#1}}
\newcommand{\Out}[1]{\operator{Out}{#1}}

%------------------
%---Теория колец---
%------------------
\newcommand*{\rad}[1]{\operator{rad}{#1}}

%------------------
%---Функторы из геометрии в алгебру---
%------------------

\newcommand*{\Isom}[1]{\operator{Isom}{#1}}
\newcommand*{\Conf}[1]{\operator{Conf}{#1}}


%------------------
%---Общая топология---
%------------------

\newcommand*{\Cl}[1]{\operator{Cl}{#1}}
\newcommand*{\Int}[1]{\operator{Int}{#1}}

%выпуклая геометрия
\newcommand*{\Conv}[1]{\operator{Conv}{#1}}
\newcommand*{\ConvInt}[1]{\operator{ConvInt}{#1}}


%------------------
%---Теория гомотопий---
%------------------


%------------------
%---Теория многообразий---
%------------------
\newcommand{\bound}[1]{\operatorTogether{\partial}{#1}}

%------------------
%---Числовые множества---
%------------------
\newcommand*{\naturalNumbers}{\mathbb{N}}
\newcommand*{\natNum}{\naturalNumbers}
\newcommand*{\N}{\naturalNumbers}

\newcommand*{\integerNumbers}{\mathbb{Z}}
\newcommand*{\intNum}{\integerNumbers}
\newcommand*{\Z}{\integerNumbers}

\newcommand*{\rationalNumbers}{\mathbb{Q}}
\newcommand*{\ratNum}{\rationalNumbers}
\newcommand*{\Q}{\rationalNumbers}


\newcommand*{\realNumbers}{\mathbb{R}}
\newcommand*{\realNum}{\realNumbers}
\newcommand*{\R}{\realNumbers}

\newcommand*{\complexNumbers}{\mathbb{C}}
\newcommand*{\compNum}{\complexNumbers}
%\newcommand*{\C}{\complexNumbers} %https://tex.stackexchange.com/questions/2967/what-is-c-with-xelatex-and-beamer

\makeatletter
\DeclareFontFamily{U}{tipa}{}
\DeclareFontShape{U}{tipa}{m}{n}{<->tipa10}{}
\newcommand{\arc@char}{{\usefont{U}{tipa}{m}{n}\symbol{62}}}%

\newcommand{\arc}[1]{\mathpalette\arc@arc{#1}}

\newcommand{\arc@arc}[2]{%
  \sbox0{$\m@th#1#2$}%
  \vbox{
    \hbox{\resizebox{\wd0}{\height}{\arc@char}}
    \nointerlineskip
    \box0
  }%
}
\makeatother
